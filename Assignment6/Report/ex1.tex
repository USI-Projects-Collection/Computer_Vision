\begin{enumerate}
  \item \emph{Rewrite as a quadratic form.}  
    Set
    \[
      X = \begin{pmatrix}x\\y\end{pmatrix}, 
      \quad
      A = \begin{pmatrix}a & b/2\\[3pt]b/2 & c\end{pmatrix}, 
      \quad
      v = \begin{pmatrix}d/2\\[3pt]e/2\end{pmatrix}.
    \]
    Then
    \[
      a x^2 + bxy + c y^2 + d x + e y + f
      = X^T A\,X + 2\,v^T X + f,
    \]
    and
    \[
      C = \begin{pmatrix} A & v \\[3pt] v^T & f \end{pmatrix}
      \;\Longrightarrow\;
      \det C = \det A \,\bigl(f - v^T A^{-1}v\bigr).
    \]

  \item \emph{Definiteness of \(A\).}  
    Since \(\delta = b^2 - 4ac < 0\), we have
    \[
      \det A = ac - \tfrac{b^2}{4} = \frac{4ac - b^2}{4} > 0.
    \]
    Moreover \(\operatorname{tr} A = a + c\).  Hence a real symmetric \(2\times2\) matrix with positive determinant has two real eigenvalues of the same sign:
    \[
      \begin{cases}
        a + c > 0 \;\Longrightarrow\;A\text{ is positive‐definite},\\
        a + c < 0 \;\Longrightarrow\;A\text{ is negative‐definite}.
      \end{cases}
    \]

  \item \emph{Complete the square.}  
    Define
    \[
      Q(X) = X^T A\,X + 2\,v^T X + f.
    \]
    Its unique critical point is
    \[
      \nabla Q = 2A\,X + 2v = 0
      \quad\Longrightarrow\quad
      X_0 = -A^{-1}v.
    \]
    Substituting back,
    \[
      Q(X) = (X - X_0)^T A\,(X - X_0)
      \;+\;\underbrace{\bigl(f - v^T A^{-1}v\bigr)}_{S},
    \]
    and one checks
    \(\det C = \det A\cdot S\), so
    \[
      S = \frac{\det C}{\det A}.
    \]

  \item \emph{Sign analysis and conclusion.}  
    We know \(\det A>0\).  The hypothesis \((a+c)\det C>0\) forces:
    \[
      \begin{cases}
        a+c>0 \implies \det C>0 \implies S>0
        &\text{and }A\text{ positive‐definite},\\
        a+c<0 \implies \det C<0 \implies S<0
        &\text{and }A\text{ negative‐definite}.
      \end{cases}
    \]
    In the first case,
    \[
      Q(X) \;=\;(X-X_0)^T A (X-X_0) + S\;\ge S>0
      \quad\forall X,
    \]
    and in the second,
    \[
      Q(X) \;=\;(X-X_0)^T A (X-X_0) + S\;\le S<0
      \quad\forall X.
    \]
    Thus in \emph{either} case \(Q(X)\) never vanishes on \(\mathbf R^2\), so
    \(\{Q=0\} = \emptyset\).  Hence the conic is \emph{imaginary}. 
\end{enumerate}