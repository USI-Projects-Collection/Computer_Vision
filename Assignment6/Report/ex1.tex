
\subsection*{1. Separate the purely quadratic part}

Write the polynomial in vector form
\[
F(x,y)=\underbrace{\begin{bmatrix}x&y\end{bmatrix}
A\begin{bmatrix}x\\y\end{bmatrix}}_{\text{quadratic}}
\;+\;
2g^{\!T}\begin{bmatrix}x\\y\end{bmatrix}+f,
\qquad
A=\begin{pmatrix}a&\tfrac{b}{2}\\[2pt]\tfrac{b}{2}&c\end{pmatrix},\;
g=\frac{1}{2}\begin{bmatrix}d\\ e\end{bmatrix}.
\]

\subsection*{2. Definiteness of $A$}

Because $\delta=b^{2}-4ac<0$,
\[
\det A
=a\,c-\Bigl(\tfrac{b}{2}\Bigr)^{2}
=\frac{-\delta}{4}>0.
\]

A symmetric $2\times2$ matrix with positive determinant is \textbf{definite}; its sign is the sign of its trace:
\begin{itemize}
\item If $a+c>0 \Rightarrow A$ is \textbf{positive definite}.
\item If $a+c<0 \Rightarrow A$ is \textbf{negative definite}.
\end{itemize}

\subsection*{3. Translate to the centre of the conic}

The gradient of $F$ vanishes at
\[
\mathbf{x}_{0}=-A^{-1}g.
\]

Completing the square gives the standard form
\[
F(x,y)=
(\mathbf{x}-\mathbf{x}_{0})^{\!T}A(\mathbf{x}-\mathbf{x}_{0})
+\Bigl(f-g^{T}A^{-1}g\Bigr).
\tag{$\star$}
\]

\subsection*{4. Express the constant term via $\det C$}

For a block matrix
$C=\begin{pmatrix}A & g \\ g^{T} & f\end{pmatrix}$
the Schur complement formula gives
\[
\boxed{\; \det C =\det A\;\Bigl(f-g^{T}A^{-1}g\Bigr) \;} \quad\Longrightarrow\quad f-g^{T}A^{-1}g=\frac{\det C}{\det A}.
\]

Insert this into ($\star$):
\[
F(x,y)=
(\mathbf{x}-\mathbf{x}_{0})^{\!T}A(\mathbf{x}-\mathbf{x}_{0})
\;+\;\frac{\det C}{\det A}.
\tag{$\star\star$}
\]

\subsection*{5. Compare the two terms in ($\star\star$)}

Recall
\begin{itemize}
\item $\det A>0$ (Step 2),
\item $A$ is definite with the same sign as $a+c$,
\item assumption $(a+c)\det C>0$ says $\det C$ \textbf{has the same sign as} $A$.
\end{itemize}

\subsubsection*{Case 1: $A$ positive definite $(a+c>0)$}
\begin{itemize}
\item $\det C>0 \Rightarrow \frac{\det C}{\det A}>0$.
\item The first term $(\mathbf{x}-\mathbf{x}_{0})^{T}A(\mathbf{x}-\mathbf{x}_{0})$ is \textbf{non-negative}, vanishing only at $\mathbf{x}=\mathbf{x}_{0}$.
\end{itemize}

Hence $F(x,y)\ge\frac{\det C}{\det A}>0$ for every $(x,y)\in\mathbb{R}^{2}$; the equation $F=0$ has \textbf{no real solution}.

\subsubsection*{Case 2: $A$ negative definite $(a+c<0)$}
\begin{itemize}
\item $\det C<0 \Rightarrow \frac{\det C}{\det A}<0$.
\item The quadratic term is \textbf{non-positive}, again zero only at $\mathbf{x}_{0}$.
\end{itemize}

Thus $F(x,y)\le\frac{\det C}{\det A}<0$ for all $(x,y)$; the equation $F=0$ is likewise \textbf{unsatisfiable}.

\subsection*{6. Conclusion}

In both cases the sign of the constant term in ($\star\star$) matches the sign of the definite quadratic part, so their sum \textbf{never vanishes} on $\mathbb{R}^{2}$.
\[
\boxed{\;
\mathcal{C}=\varnothing
\;}
\]

Therefore the conic determined by $C$ is \emph{imaginary} (has no real points) whenever
\[
\delta=b^{2}-4ac<0,\qquad (a+c)\det C>0,\qquad\det C\neq0.
\]