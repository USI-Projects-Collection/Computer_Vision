\documentclass[12pt]{article}
\usepackage[utf8]{inputenc}
\usepackage{tikz}
\usepackage[top=1in, bottom=1in]{geometry}
\usepackage{graphicx}
\usepackage{float}
\usepackage{amsmath}
\usepackage{subcaption}
\usepackage{listings}
\usepackage{hyperref}
\usepackage{amssymb}


\lstdefinestyle{pythonstyle}{
    language=Python,
    basicstyle=\ttfamily\scriptsize,
    keywordstyle=\color{blue}\bfseries,
    stringstyle=\color{red},
    commentstyle=\color{gray},
    backgroundcolor=\color{white},
    frame=single,
    rulecolor=\color{black},
    breaklines=true,
    numbers=left,
    numberstyle=\tiny\color{gray},
    captionpos=b,
    tabsize=4,
    showstringspaces=false
}
\lstset{style=pythonstyle}

\author{
	Paolo Deidda (\text{paolo.deidda@usi.ch}) \\ 
    Raffaele Perri (\text{raffaele.perri@usi.ch}) \\
    \url{https://github.com/USI-Projects-Collection/Computer_Vision.git}
}

\date{\today}

\begin{document}


% ===============================
\title{Assignment 7}
\maketitle

\section*{Problem 1 [10 points]}
In the projective plane \( \mathbb{P}^2 \), a line is represented by a homogeneous vector \( (a, b, c)^T \), defining the equation \( ax + by + cz = 0 \), where \( (x, y, z)^T \) are the homogeneous coordinates of points on the line. The intersection of two lines is found using the cross product of their homogeneous vectors.

\subsection*{Step 1: Define the Line Equations}
\begin{itemize}
    \item For line \( l = (2, 4, 8)^T \), the equation is:
    \[
    2x + 4y + 8z = 0
    \]
    \item For line \( m = (-2, 3, -1)^T \), the equation is:
    \[
    -2x + 3y - z = 0
    \]
\end{itemize}

\subsection*{Step 2: Compute the Cross Product}
The homogeneous coordinates of the intersection point are given by \( p = l \times m \). For vectors \( l = (l_1, l_2, l_3) = (2, 4, 8) \) and \( m = (m_1, m_2, m_3) = (-2, 3, -1) \), the components are:
\begin{align*}
    p_x &= l_2 m_3 - l_3 m_2 = 4 \cdot (-1) - 8 \cdot 3 = -4 - 24 = -28 \\
    p_y &= l_3 m_1 - l_1 m_3 = 8 \cdot (-2) - 2 \cdot (-1) = -16 - (-2) = -16 + 2 = -14 \\
    p_z &= l_1 m_2 - l_2 m_1 = 2 \cdot 3 - 4 \cdot (-2) = 6 - (-8) = 6 + 8 = 14
\end{align*}
Thus, the homogeneous coordinates are:
\[
p = (-28, -14, 14)^T
\]
Since homogeneous coordinates are defined up to scale, divide by 14:
\[
p = \left( \frac{-28}{14}, \frac{-14}{14}, \frac{14}{14} \right) = (-2, -1, 1)^T
\]

\subsection*{Step 3: Convert to Cartesian Coordinates}
For a point \( (x, y, z)^T \) with \( z \neq 0 \), the Cartesian coordinates are \( (x/z, y/z) \). Here, \( z = 1 \), so:
\[
x = \frac{-2}{1} = -2, \quad y = \frac{-1}{1} = -1
\]
Thus, the Cartesian coordinates are \( (-2, -1) \).

\subsection*{Step 4: Verification}
Substitute \( (x, y, z) = (-2, -1, 1) \) into both equations:
\begin{itemize}
    \item Line \( l \): \( 2(-2) + 4(-1) + 8(1) = -4 - 4 + 8 = 0 \)
    \item Line \( m \): \( -2(-2) + 3(-1) - 1 = 4 - 3 - 1 = 0 \)
\end{itemize}
Both are satisfied, confirming the solution.

\subsection*{Final Answer}
The Cartesian coordinates of the intersection point are:
\[
\boxed{(-2, -1)}
\]

% \section*{Problem 2 [5 points]}
% The implementation follows these steps:
\begin{enumerate}
    \item \textbf{Image Preprocessing:} 
    \begin{itemize}
        \item Load the input image.
        \item Convert it to grayscale if necessary.
    \end{itemize}
    \item \textbf{Applying Otsu’s Thresholding:}
    \begin{itemize}
        \item Compute the optimal threshold automatically using Otsu’s method.
        \item Apply the threshold to generate a binary image.
    \end{itemize}
    \item \textbf{Visualization and Output:}
    \begin{itemize}
        \item Display both the original and thresholded images.
        \item Save the resulting binary image.
    \end{itemize}
\end{enumerate}

The function \texttt{otsu\_thresholding()} converts the image to grayscale if it is not already and then applies Otsu’s thresholding using \texttt{cv2.threshold()}. \\
The key component of the function is:
\begin{verbatim}
    _, binary_image = cv2.threshold(gray_image, 0, 255, cv2.THRESH_BINARY +
    cv2.THRESH_OTSU)
\end{verbatim}

Here, Otsu’s method automatically determines the threshold value by analyzing the image’s histogram and maximizing the inter-class variance. \\
Applying Otsu’s method to "houses.pgm" effectively separates the objects in the image from the background. The resulting binary image highlights structural details such as edges of buildings and other prominent objects.
\begin{figure}[H]
    \centering
    \includegraphics[width=\linewidth]{Assignment4/Images/otsu_thresholding-2.png}
    \caption{Otsu method.}
    \label{fig:enter-label}
\end{figure}

The implementation successfully applies Otsu’s method to "houses.pgm," demonstrating its ability to segment objects efficiently.
\begin{figure}[H]
    \centering
    \includegraphics[width=0.65\linewidth]{Assignment4/Images/binary_image.png}
    \caption{Final image.}
    \label{fig:enter-label}
\end{figure}
Future improvements could involve adaptive thresholding techniques for cases where global thresholding is insufficient due to uneven lighting conditions.

% \section*{Problem 3 [5 points]}
% The code is provided in the separate file \textbf{source.py}.

\begin{figure}[H]
    \centering
    \includegraphics[width=0.4\textwidth]{../Assets/Case_1.png}
    \includegraphics[width=0.4\textwidth]{../Assets/Case_2.png}
    \includegraphics[width=0.4\textwidth]{../Assets/Case_3.png}
    \caption{Intersection of lines \( l \) and \( m \)}
    \label{fig:intersection}
\end{figure}
% ===============================


\end{document}